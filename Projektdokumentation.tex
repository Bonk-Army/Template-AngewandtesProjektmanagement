%======================================================================
%	
%	 __  __       _         ______ _ _      
%	|  \/  |     (_)       |  ____(_) |     
%	| \  / | __ _ _ _ __   | |__   _| | ___ 
%	| |\/| |/ _` | | '_ \  |  __| | | |/ _ \
%	| |  | | (_| | | | | | | |    | | |  __/
%	|_|  |_|\__,_|_|_| |_| |_|    |_|_|\___|
%
%----------------------------------------------------------------------
% Descripton : Main File to compose the resulting PDF
%======================================================================
%\pdfobjcompresslevel=0
\pdfminorversion=7
\pdfinclusioncopyfonts=1
\documentclass[
				ngerman             % neue deutsche Rechtschreibung
				,a4paper            % Papiergrösse
				,twoside            % Zweiseitiger Druck (rechts/links)
				% ,10pt             % Schriftgrösse
				,11pt
				% ,12pt
				,pdftex
				%  ,disable         % Todo-Markierungen auschalten
				,notitlepage		% Thanks abstract... preventing it from removing pageNumber
				]{report}

% Bitte die Codierung Ihrer Dateien auswählen:
% \usepackage[latin1]{inputenc}    % Für UNIX mit ISO-LATIN-codierten Dateien
% \usepackage[applemac]{inputenc}  % Für Apple Mac
% \usepackage[ansinew]{inputenc}   % Für Microsoft Windows
\usepackage[utf8]{inputenc}        % UTF-8 codierte Dateien
% Dieses Dokument ist unter Unix erstellt, daher
% wird diese Input-Codierung benutzt.

%————————————————————————————————————————————————————————————————————————————
%					Import aller Konfigurationen
%————————————————————————————————————————————————————————————————————————————
\usepackage{./config/DHBW/bericht}
\usepackage{./config/PaketeProjektarbeit} % <- spezifische Pakete für die Projektarbeit
\usepackage{./config/Pakete}
\usepackage{./config/Befehle}
\usepackage{./config/Layout}
\usepackage{./config/Styles}
%Wenn PDF/A-1B gewünscht .. Achtung am besten nochmal mit veraPDF testen!
\usepackage{./config/PDFStandard}

%————————————————————————————————————————————————————————————————————————————
%					Variablen deklarieren
%————————————————————————————————————————————————————————————————————————————
\include{./config/Variables}

%————————————————————————————————————————————————————————————————————————————
%					Informationen ausfüllen
%————————————————————————————————————————————————————————————————————————————
%======================================================================
%	
%	_____        __                           _   _             
%  |_   _|      / _|                         | | (_)            
%	 | |  _ __ | |_ ___  _ __ _ __ ___   __ _| |_ _  ___  _ __  
%    | | | '_ \|  _/ _ \| '__| '_ ` _ \ / _` | __| |/ _ \| '_ \ 
%	_| |_| | | | || (_) | |  | | | | | | (_| | |_| | (_) | | | |
%  |_____|_| |_|_| \___/|_|  |_| |_| |_|\__,_|\__|_|\___/|_| |_|
%
%----------------------------------------------------------------------
% Descripton : File with all Information about the Document
%======================================================================

% Bei jedem Dokument Anpassen
\newcommand{\Was}{HEHEHHEE}

%%%%%%%%%%%%%%%%%%%%%%%%%%%%%%%%%%%%%%%%%%%%%%%%%%%%%%%%%%%%%%%%%%%%%%%%%%%%%%%%%%%%%

\newcommand{\Titel}{Projektgruppe : XXXXXX}
\newcommand{\Autor}{Max Mustermann}
\newcommand{\AutorTwo}{Max Mustermann}
\newcommand{\AutorThree}{Max Mustermann}
\newcommand{\VorlesungsTitel}{Angewandtes Projektmanagament}
\newcommand{\Kursbezeichnung}{tinf17b3}

% Falls es kein Firmenlogo gibt:
\newcommand{\FirmenLogoDeckblatt}{}

\newcommand{\BetreuerDHBW}{Michael Vetter}

%%%%%%%%%%%%%%%%%%%%%%%%%%%%%%%%%%%%%%%%%%%%%%%%%%%%%%%%%%%%%%%%%%%%%%%%%%%%%%%%%%%%%


\newcommand{\AbgabeDatum}{1. April 2090}

\newcommand{\Dauer}{12 Wochen}

% \newcommand{\Abschluss}{Bachelor of Engineering}
\newcommand{\Abschluss}{Bachelor of Science}

\newcommand{\Studiengang}{Informatik / Informationstechnik}
% \newcommand{\Studiengang}{Informatik / Angewandte Informatik}
\include{./config/HyperrefSetup}


%————————————————————————————————————————————————————————————————————————————
%					Sonstige Einstellungen
%————————————————————————————————————————————————————————————————————————————
\bibliography{./directories/bibliography}
\loadglsentries{./directories/glossary}
\makeglossaries
\makeindex[columns=1,options={-s ./config/DHBW/bericht.ist}]

%————————————————————————————————————————————————————————————————————————————
%					Einstellungen
%————————————————————————————————————————————————————————————————————————————
% Debug auf true setzen, hierbei wird der Changelog hinzugefügt, und keine Doppelseitenumbrüche gemacht (Zum drucken -> false)
% Bitte zum drucken oben in der Präambel auch auf Doppelseite schalten!
\setboolean{DEBUG}{true}
% Sollen die TODO's angezeigt werden ?
\setboolean{TODO}{true}
% =================================================================================================
% Soll ein Index eingefügt werden ?
\setboolean{INDEX}{true}
%Soll ne Trennlinie im Inhaltsverzeichnis eingefügt werden, vor Verzeichnissen, dem Inhalt und dem Anhang
\setboolean{TRENNSTRICHE}{true}
% =================================================================================================

%————————————————————————————————————————————————————————————————————————————
%					PDF Optionen
%————————————————————————————————————————————————————————————————————————————
%Soll der Changelog angezeigt werden -- Sowieso nur im DEBUG?
\ifthenelse{\boolean{DEBUG}}{}{\setboolean{CHANGELOG}{false}}
%Package um die Referenzen zu checken, einfach im Log nach "refcheck suchen!"
%\usepackage{refcheck}

%————————————————————————————————————————————————————————————————————————————
%									Anfang Dokument
%————————————————————————————————————————————————————————————————————————————
\begin{document}
	% Wer keine Seitenzahlen braucht bis zum TOC einfach das Pagenumbering auf gobble und an der Stelle an der die Seitenzählung beginnen soll roman einfügen
	\pagenumbering{gobble}
	\pagestyle{plain}
	
	%—————————————————————————————————————————————————————————————————————————
	%								TODOS
	%—————————————————————————————————————————————————————————————————————————
	\ifthenelse{\boolean{TODO}}{
		\phantomsection
		\addcontentsline{toc}{chapter}{Liste der ToDo's}
		\listoftodos[Liste der ToDo's]
		\clearpage
		\ifthenelse{\boolean{TRENNSTRICHE}}{\addtocontents{toc}{\protect\mbox{}\protect\hrulefill\par}}{}
	}{}
	
	%—————————————————————————————————————————————————————————————————————————
	%								Deckseite
	%—————————————————————————————————————————————————————————————————————————
	\begin{titlepage}
		%%%%%%%%%%%%%%%%%%%%%%%%%%%%%%%%%%%%%%%%%%%%%%%%%%%%%%%%%%%%%%%%%%%%%%%%%%%%%%%
%% Descr:       Vorlage für Berichte der DHBW-Karlsruhe, Titlepage
%% Author:      Prof. Dr. Jürgen Vollmer, vollmer@dhbw-karlsruhe.de
%% $Id: erklaerung.tex,v 1.11 2020/03/13 14:24:42 vollmer Exp $
%% -*- coding: utf-8 -*-
%%%%%%%%%%%%%%%%%%%%%%%%%%%%%%%%%%%%%%%%%%%%%%%%%%%%%%%%%%%%%%%%%%%%%%%%%%%%%%%

\begin{center}
	\vspace*{-2cm}
	\FirmenLogoDeckblatt\hfill\includegraphics[width=4cm]{./config/DHBW/dhbw-logo.png}\\[2cm]
	{\Huge \Titel}\\[1cm]
	{\Huge\scshape \Was}\\[1cm]
	{\Large für die Vorlesung}\\[0.5cm]
	{\Large \VorlesungsTitel}\\[0.5cm]
	{\large des Studienganges \Studiengang}\\[0.5cm]
	{\large an der}\\[0.5cm]
	{\large Dualen Hochschule Baden-Württemberg Karlsruhe}\\[0.5cm]
	{\large von}\\[0.5cm]
	{\large\bfseries \Autor \\ \AutorTwo \\ \AutorThree}\\[1cm]
	{\large Abgabedatum des Dokumentes \underline{\today}}
	\vfill
\end{center}
\begin{tabular}{l@{\hspace{2cm}}l}
	Kurs			         			& \Kursbezeichnung			\\
	Gutachter der Studienakademie	 	& \BetreuerDHBW				\\
	Betreuer der Studienarbeit		 	& \BetreuerStudienarbeit	\\
\end{tabular}
	\end{titlepage}
	
	% Ab hier beginnt die Seitenzählung für den Prefix in kleinen römischen Zahlen
	% Erste Seite nach der Titelseite
	\pagenumbering{roman}
	\setcounter{page}{2}
	\ifthenelse{\boolean{DEBUG}}{}{\cleardoublepage}
	
	%—————————————————————————————————————————————————————————————————————————
	%							Inhaltsverzeichnis
	%—————————————————————————————————————————————————————————————————————————
	\setcounter{tocdepth}{1}% Allow only \chapter in ToC
	\ifthenelse{\boolean{DEBUG}}{}{\cleardoublepage}
	
	\phantomsection
	\addcontentsline{toc}{chapter}{Inhaltsverzeichnis}
	\tableofcontents           % Inhaltsverzeichnis hier ausgeben
	\ifthenelse{\boolean{DEBUG}}{\newpage}{\cleardoublepage}
	
	%\phantomsection
	%\listoffigures             % Liste der Abbildungen
	%\addcontentsline{toc}{chapter}{Abbildungsverzeichnis}
	%\ifthenelse{\boolean{DEBUG}}{\newpage}{\cleardoublepage}
	
	\phantomsection
	\include{directories/acronyms} %Acronyms
	\ifthenelse{\boolean{DEBUG}}{\newpage}{\cleardoublepage}
	
	\phantomsection
	%\addcontentsline{toc}{chapter}{Glossar} %Glossar
	% Wenn Glossar nicht erstellt wird -> In den Projektarbeitspaketen automake=true setzen ....
	\printglossaries
	\glsaddall
	
	\ifthenelse{\boolean{TRENNSTRICHE}}{\addtocontents{toc}{\protect\mbox{}\protect\hrulefill\par}}{}
	
	\ifthenelse{\boolean{DEBUG}}{\newpage}{\cleardoublepage}
	
	%—————————————————————————————————————————————————————————————————————————
	%								Inhalt
	%—————————————————————————————————————————————————————————————————————————
	%Hauptteil normale arabische Zählung, beginnend bei 1 und alten Pagestyle wiederherstellen!
	\pagenumbering{arabic} 
	\setcounter{page}{1}
	\pagestyle{headings} %<- Pagestyle reset 
	
	% Einschub für die Trennlinie im ToC
	\ifthenelse{\boolean{TRENNSTRICHE}}{\addtocontents{toc}{\protect\mbox{}\protect\hrulefill\par} \clearpage}{}
	
	% Hier entsprechend EINEN Teile auskommentieren und vor allem TITELPAGEINFORMATIONEN IMMER anpassen!!!
	% BITTE auch den Appendix richtig auskommentieren
	% !TeX spellcheck = de_DE
%%%%%%%%%%%%%%%%%%%%%%%%%%%%%%%%%%%%%%%%%%%%%%%%%%%%%%%%%%%%%%%%%%%%%%%%%%%%%%%%%%%%%%%%%
%	 _____           _      _    _   _     _           
%	|  __ \         (_)    | |  | | (_)   | |          
%	| |__) | __ ___  _  ___| | _| |_ _  __| | ___  ___ 
%	|  ___/ '__/ _ \| |/ _ \ |/ / __| |/ _` |/ _ \/ _ \
%	| |   | | | (_) | |  __/   <| |_| | (_| |  __/  __/
%	|_|   |_|  \___/| |\___|_|\_\\__|_|\__,_|\___|\___|
%				   _/ |                                
%				  |__/                                 
%%%%%%%%%%%%%%%%%%%%%%%%%%%%%%%%%%%%%%%%%%%%%%%%%%%%%%%%%%%%%%%%%%%%%%%%%%%%%%%%%%%%%%%%%
% .....
%%%%%%%%%%%%%%%%%%%%%%%%%%%%%%%%%%%%%%%%%%%%%%%%%%%%%%%%%%%%%%%%%%%%%%%%%%%%%%%%%%%%%%%%%

\chapter{Noch keine Chaptereinteilung vorhanden!}

	% !TeX spellcheck = de_DE
%%%%%%%%%%%%%%%%%%%%%%%%%%%%%%%%%%%%%%%%%%%%%%%%%%%%%%%%%%%%%%%%%%%%%%%%%%%%%%%%%%%%%%%%%
%	 _____           _      _    _   _                             _   
%	|  __ \         (_)    | |  | | | |                           | |  
%	| |__) | __ ___  _  ___| | _| |_| | _____  _ __  _______ _ __ | |_ 
%	|  ___/ '__/ _ \| |/ _ \ |/ / __| |/ / _ \| '_ \|_  / _ \ '_ \| __|
%	| |   | | | (_) | |  __/   <| |_|   < (_) | | | |/ /  __/ |_) | |_ 
%	|_|   |_|  \___/| |\___|_|\_\\__|_|\_\___/|_| |_/___\___| .__/ \__|
%				   _/ |                                     | |        
%				  |__/                                      |_|                   
%%%%%%%%%%%%%%%%%%%%%%%%%%%%%%%%%%%%%%%%%%%%%%%%%%%%%%%%%%%%%%%%%%%%%%%%%%%%%%%%%%%%%%%%%
% .....
%%%%%%%%%%%%%%%%%%%%%%%%%%%%%%%%%%%%%%%%%%%%%%%%%%%%%%%%%%%%%%%%%%%%%%%%%%%%%%%%%%%%%%%%%

\chapter{Ausgangslage}

\chapter{Ansatz für Implementierung}

\chapter{Zusammenfassung}

\chapter{Alternative Ansätze}
	% !TeX spellcheck = de_DE
%%%%%%%%%%%%%%%%%%%%%%%%%%%%%%%%%%%%%%%%%%%%%%%%%%%%%%%%%%%%%%%%%%%%%%%%%%%%%%%%%%%%%%%%%
%	 _____           _      _       _       __ _       _ _   _             
%	|  __ \         (_)    | |     | |     / _(_)     (_) | (_)            
%	| |__) | __ ___  _  ___| | ____| | ___| |_ _ _ __  _| |_ _  ___  _ __  
%	|  ___/ '__/ _ \| |/ _ \ |/ / _` |/ _ \  _| | '_ \| | __| |/ _ \| '_ \ 
%	| |   | | | (_) | |  __/   < (_| |  __/ | | | | | | | |_| | (_) | | | |
%	|_|   |_|  \___/| |\___|_|\_\__,_|\___|_| |_|_| |_|_|\__|_|\___/|_| |_|
%				   _/ |                                                    
%				  |__/                                                                  
%%%%%%%%%%%%%%%%%%%%%%%%%%%%%%%%%%%%%%%%%%%%%%%%%%%%%%%%%%%%%%%%%%%%%%%%%%%%%%%%%%%%%%%%%
% .....
%%%%%%%%%%%%%%%%%%%%%%%%%%%%%%%%%%%%%%%%%%%%%%%%%%%%%%%%%%%%%%%%%%%%%%%%%%%%%%%%%%%%%%%%%

\chapter{Projektdefinition}
	
	\section{Funktionsumfang}
	
	\section{Organisation des Projektes}
	
	\section{Risiken}
	% !TeX spellcheck = de_DE
%%%%%%%%%%%%%%%%%%%%%%%%%%%%%%%%%%%%%%%%%%%%%%%%%%%%%%%%%%%%%%%%%%%%%%%%%%%%%%%%%%%%%%%%%
%	 _____           _      _    _        _     _              __ 
%	|  __ \         (_)    | |  | |      | |   | |            / _|
%	| |__) | __ ___  _  ___| | _| |_ __ _| |__ | | __ _ _   _| |_ 
%	|  ___/ '__/ _ \| |/ _ \ |/ / __/ _` | '_ \| |/ _` | | | |  _|
%	| |   | | | (_) | |  __/   <| || (_| | |_) | | (_| | |_| | |  
%	|_|   |_|  \___/| |\___|_|\_\\__\__,_|_.__/|_|\__,_|\__,_|_|  
%				   _/ |                                           
%				   |__/                                                              
%%%%%%%%%%%%%%%%%%%%%%%%%%%%%%%%%%%%%%%%%%%%%%%%%%%%%%%%%%%%%%%%%%%%%%%%%%%%%%%%%%%%%%%%%
% Strukturplanung
% Aufwandschätzung
% Netzplantechnik
% Arbeitsplanung
% Kostenplanung
% Risikomanagement
% Projektpläne
%%%%%%%%%%%%%%%%%%%%%%%%%%%%%%%%%%%%%%%%%%%%%%%%%%%%%%%%%%%%%%%%%%%%%%%%%%%%%%%%%%%%%%%%%

\chapter{Strukturplanung}

	\section{Terminplan}
	
		\subsection{Meilensteine}
	
	\section{Strukturplanung}
	
	\section{QS-Maßnahmen}
	
	\section{Aufwandsschätzung}
	% !TeX spellcheck = de_DE
%%%%%%%%%%%%%%%%%%%%%%%%%%%%%%%%%%%%%%%%%%%%%%%%%%%%%%%%%%%%%%%%%%%%%%%%%%%%%%%%%%%%%%%%%
%	 _____  _     _ _                           _                
%	|  __ \(_)   (_) |                         | |               
%	| |__) |_ ___ _| | _____   __ _ _ __   __ _| |_   _ ___  ___ 
%	|  _  /| / __| | |/ / _ \ / _` | '_ \ / _` | | | | / __|/ _ \
%	| | \ \| \__ \ |   < (_) | (_| | | | | (_| | | |_| \__ \  __/
%	|_|  \_\_|___/_|_|\_\___/ \__,_|_| |_|\__,_|_|\__, |___/\___|
%												   __/ |         
%												  |___/                        
%%%%%%%%%%%%%%%%%%%%%%%%%%%%%%%%%%%%%%%%%%%%%%%%%%%%%%%%%%%%%%%%%%%%%%%%%%%%%%%%%%%%%%%%%
% .....
%%%%%%%%%%%%%%%%%%%%%%%%%%%%%%%%%%%%%%%%%%%%%%%%%%%%%%%%%%%%%%%%%%%%%%%%%%%%%%%%%%%%%%%%%

\chapter{Noch keine Chaptereinteilung vorhanden!}
	% !TeX spellcheck = de_DE
%%%%%%%%%%%%%%%%%%%%%%%%%%%%%%%%%%%%%%%%%%%%%%%%%%%%%%%%%%%%%%%%%%%%%%%%%%%%%%%%%%%%%%%%%
%	  ____   _____       __  __                           _                          
%	 / __ \ / ____|     |  \/  |                         | |                         
%	| |  | | (___ ______| \  / | __ _ ___ ___ _ __   __ _| |__  _ __ ___   ___ _ __  
%	| |  | |\___ \______| |\/| |/ _` / __/ __| '_ \ / _` | '_ \| '_ ` _ \ / _ \ '_ \ 
%	| |__| |____) |     | |  | | (_| \__ \__ \ | | | (_| | | | | | | | | |  __/ | | |
%	 \___\_\_____/      |_|  |_|\__,_|___/___/_| |_|\__,_|_| |_|_| |_| |_|\___|_| |_|
%	          
%%%%%%%%%%%%%%%%%%%%%%%%%%%%%%%%%%%%%%%%%%%%%%%%%%%%%%%%%%%%%%%%%%%%%%%%%%%%%%%%%%%%%%%%%
% .....
%%%%%%%%%%%%%%%%%%%%%%%%%%%%%%%%%%%%%%%%%%%%%%%%%%%%%%%%%%%%%%%%%%%%%%%%%%%%%%%%%%%%%%%%%

	% !TeX spellcheck = de_DE
%%%%%%%%%%%%%%%%%%%%%%%%%%%%%%%%%%%%%%%%%%%%%%%%%%%%%%%%%%%%%%%%%%%%%%%%%%%%%%%%%%%%%%%%%
%	 _____           _      _    _        _              _     _               
%	|  __ \         (_)    | |  | |      | |            | |   | |              
%	| |__) | __ ___  _  ___| | _| |_ __ _| |__  ___  ___| |__ | |_   _ ___ ___ 
%	|  ___/ '__/ _ \| |/ _ \ |/ / __/ _` | '_ \/ __|/ __| '_ \| | | | / __/ __|
%	| |   | | | (_) | |  __/   <| || (_| | |_) \__ \ (__| | | | | |_| \__ \__ \
%	|_|   |_|  \___/| |\___|_|\_\\__\__,_|_.__/|___/\___|_| |_|_|\__,_|___/___/
%				   _/ |                                                        
%				  |__/                                                                          
%%%%%%%%%%%%%%%%%%%%%%%%%%%%%%%%%%%%%%%%%%%%%%%%%%%%%%%%%%%%%%%%%%%%%%%%%%%%%%%%%%%%%%%%%
% .....
%%%%%%%%%%%%%%%%%%%%%%%%%%%%%%%%%%%%%%%%%%%%%%%%%%%%%%%%%%%%%%%%%%%%%%%%%%%%%%%%%%%%%%%%%

\chapter{Noch keine Chaptereinteilung vorhanden!}
	% Zwischenstände
	% !TeX spellcheck = de_DE
%%%%%%%%%%%%%%%%%%%%%%%%%%%%%%%%%%%%%%%%%%%%%%%%%%%%%%%%%%%%%%%%%%%%%%%%%%%%%%%%%%%%%%%%%
%	 ______        _          _                    _                  _ 
%	|___  /       (_)        | |                  | |                | |
%	   / /_      ___ ___  ___| |__   ___ _ __  ___| |_ __ _ _ __   __| |
%	  / /\ \ /\ / / / __|/ __| '_ \ / _ \ '_ \/ __| __/ _` | '_ \ / _` |
%	 / /__\ V  V /| \__ \ (__| | | |  __/ | | \__ \ || (_| | | | | (_| |
%	/_____|\_/\_/ |_|___/\___|_| |_|\___|_| |_|___/\__\__,_|_| |_|\__,_|
%         
%%%%%%%%%%%%%%%%%%%%%%%%%%%%%%%%%%%%%%%%%%%%%%%%%%%%%%%%%%%%%%%%%%%%%%%%%%%%%%%%%%%%%%%%%
% .....
%%%%%%%%%%%%%%%%%%%%%%%%%%%%%%%%%%%%%%%%%%%%%%%%%%%%%%%%%%%%%%%%%%%%%%%%%%%%%%%%%%%%%%%%%

\chapter{Noch keine Chaptereinteilung vorhanden!}
	
	\ifthenelse{\boolean{DEBUG}}{}{\cleardoublepage}
	
	%—————————————————————————————————————————————————————————————————————————
	%								Anhangs
	%—————————————————————————————————————————————————————————————————————————
	%Anhang mit großen römischen Zahlen
	\pagenumbering{Roman}
	\appendix
	%======================================================================
%	
%	                _                       
%	    /\         | |                      
%	   /  \   _ __ | |__   __ _ _ __   __ _ 
%	  / /\ \ | '_ \| '_ \ / _` | '_ \ / _` |
%	 / ____ \| | | | | | | (_| | | | | (_| |
%	/_/    \_\_| |_|_| |_|\__,_|_| |_|\__, |
%								       __/ |
%									  |___/ 
%
%----------------------------------------------------------------------
% Descripton : Appendix Main File to Compose differtend Appendix
%======================================================================

\pagestyle{plain}

\chapter{Anhang}

%Fixxing Lstlisting and formular
\addtocontents{lol}{\protect\addvspace{10pt}}
\addtocontents{for}{\protect\addvspace{10pt}}

% DIREKT HIER DEN ANHANG
% !TeX spellcheck = de_DE
%%%%%%%%%%%%%%%%%%%%%%%%%%%%%%%%%%%%%%%%%%%%%%%%%%%%%%%%%%%%%%%%%%%%%%%%%%%%%%%%%%%%%%%%%
%	 _____           _      _    _   _     _           
%	|  __ \         (_)    | |  | | (_)   | |          
%	| |__) | __ ___  _  ___| | _| |_ _  __| | ___  ___ 
%	|  ___/ '__/ _ \| |/ _ \ |/ / __| |/ _` |/ _ \/ _ \
%	| |   | | | (_) | |  __/   <| |_| | (_| |  __/  __/
%	|_|   |_|  \___/| |\___|_|\_\\__|_|\__,_|\___|\___|
%				   _/ |                                
%				  |__/                                 
%%%%%%%%%%%%%%%%%%%%%%%%%%%%%%%%%%%%%%%%%%%%%%%%%%%%%%%%%%%%%%%%%%%%%%%%%%%%%%%%%%%%%%%%%
% .....
%%%%%%%%%%%%%%%%%%%%%%%%%%%%%%%%%%%%%%%%%%%%%%%%%%%%%%%%%%%%%%%%%%%%%%%%%%%%%%%%%%%%%%%%%

\chapter{Noch keine Chaptereinteilung vorhanden!}

% !TeX spellcheck = de_DE
%%%%%%%%%%%%%%%%%%%%%%%%%%%%%%%%%%%%%%%%%%%%%%%%%%%%%%%%%%%%%%%%%%%%%%%%%%%%%%%%%%%%%%%%%
%	 _____           _      _    _   _                             _   
%	|  __ \         (_)    | |  | | | |                           | |  
%	| |__) | __ ___  _  ___| | _| |_| | _____  _ __  _______ _ __ | |_ 
%	|  ___/ '__/ _ \| |/ _ \ |/ / __| |/ / _ \| '_ \|_  / _ \ '_ \| __|
%	| |   | | | (_) | |  __/   <| |_|   < (_) | | | |/ /  __/ |_) | |_ 
%	|_|   |_|  \___/| |\___|_|\_\\__|_|\_\___/|_| |_/___\___| .__/ \__|
%				   _/ |                                     | |        
%				  |__/                                      |_|                   
%%%%%%%%%%%%%%%%%%%%%%%%%%%%%%%%%%%%%%%%%%%%%%%%%%%%%%%%%%%%%%%%%%%%%%%%%%%%%%%%%%%%%%%%%
% .....
%%%%%%%%%%%%%%%%%%%%%%%%%%%%%%%%%%%%%%%%%%%%%%%%%%%%%%%%%%%%%%%%%%%%%%%%%%%%%%%%%%%%%%%%%

\chapter{Ausgangslage}

\chapter{Ansatz für Implementierung}

\chapter{Zusammenfassung}

\chapter{Alternative Ansätze}
% !TeX spellcheck = de_DE
%%%%%%%%%%%%%%%%%%%%%%%%%%%%%%%%%%%%%%%%%%%%%%%%%%%%%%%%%%%%%%%%%%%%%%%%%%%%%%%%%%%%%%%%%
%	 _____           _      _       _       __ _       _ _   _             
%	|  __ \         (_)    | |     | |     / _(_)     (_) | (_)            
%	| |__) | __ ___  _  ___| | ____| | ___| |_ _ _ __  _| |_ _  ___  _ __  
%	|  ___/ '__/ _ \| |/ _ \ |/ / _` |/ _ \  _| | '_ \| | __| |/ _ \| '_ \ 
%	| |   | | | (_) | |  __/   < (_| |  __/ | | | | | | | |_| | (_) | | | |
%	|_|   |_|  \___/| |\___|_|\_\__,_|\___|_| |_|_| |_|_|\__|_|\___/|_| |_|
%				   _/ |                                                    
%				  |__/                                                                  
%%%%%%%%%%%%%%%%%%%%%%%%%%%%%%%%%%%%%%%%%%%%%%%%%%%%%%%%%%%%%%%%%%%%%%%%%%%%%%%%%%%%%%%%%
% .....
%%%%%%%%%%%%%%%%%%%%%%%%%%%%%%%%%%%%%%%%%%%%%%%%%%%%%%%%%%%%%%%%%%%%%%%%%%%%%%%%%%%%%%%%%

\chapter{Projektdefinition}
	
	\section{Funktionsumfang}
	
	\section{Organisation des Projektes}
	
	\section{Risiken}
% !TeX spellcheck = de_DE
%%%%%%%%%%%%%%%%%%%%%%%%%%%%%%%%%%%%%%%%%%%%%%%%%%%%%%%%%%%%%%%%%%%%%%%%%%%%%%%%%%%%%%%%%
%	 _____           _      _    _        _     _              __ 
%	|  __ \         (_)    | |  | |      | |   | |            / _|
%	| |__) | __ ___  _  ___| | _| |_ __ _| |__ | | __ _ _   _| |_ 
%	|  ___/ '__/ _ \| |/ _ \ |/ / __/ _` | '_ \| |/ _` | | | |  _|
%	| |   | | | (_) | |  __/   <| || (_| | |_) | | (_| | |_| | |  
%	|_|   |_|  \___/| |\___|_|\_\\__\__,_|_.__/|_|\__,_|\__,_|_|  
%				   _/ |                                           
%				   |__/                                                              
%%%%%%%%%%%%%%%%%%%%%%%%%%%%%%%%%%%%%%%%%%%%%%%%%%%%%%%%%%%%%%%%%%%%%%%%%%%%%%%%%%%%%%%%%
% .....
%%%%%%%%%%%%%%%%%%%%%%%%%%%%%%%%%%%%%%%%%%%%%%%%%%%%%%%%%%%%%%%%%%%%%%%%%%%%%%%%%%%%%%%%%

\chapter{Grundvoraussetzungen}

\section{Konzept}

\section{Evaluation}

\section{Entwicklung}

\section{Absprachen}

\section{Dokumentation}
% !TeX spellcheck = de_DE
%%%%%%%%%%%%%%%%%%%%%%%%%%%%%%%%%%%%%%%%%%%%%%%%%%%%%%%%%%%%%%%%%%%%%%%%%%%%%%%%%%%%%%%%%
%	 _____  _     _ _                           _                
%	|  __ \(_)   (_) |                         | |               
%	| |__) |_ ___ _| | _____   __ _ _ __   __ _| |_   _ ___  ___ 
%	|  _  /| / __| | |/ / _ \ / _` | '_ \ / _` | | | | / __|/ _ \
%	| | \ \| \__ \ |   < (_) | (_| | | | | (_| | | |_| \__ \  __/
%	|_|  \_\_|___/_|_|\_\___/ \__,_|_| |_|\__,_|_|\__, |___/\___|
%												   __/ |         
%												  |___/                        
%%%%%%%%%%%%%%%%%%%%%%%%%%%%%%%%%%%%%%%%%%%%%%%%%%%%%%%%%%%%%%%%%%%%%%%%%%%%%%%%%%%%%%%%%
% .....
%%%%%%%%%%%%%%%%%%%%%%%%%%%%%%%%%%%%%%%%%%%%%%%%%%%%%%%%%%%%%%%%%%%%%%%%%%%%%%%%%%%%%%%%%

\chapter{Noch keine Chaptereinteilung vorhanden!}
% !TeX spellcheck = de_DE
%%%%%%%%%%%%%%%%%%%%%%%%%%%%%%%%%%%%%%%%%%%%%%%%%%%%%%%%%%%%%%%%%%%%%%%%%%%%%%%%%%%%%%%%%
%	  ____   _____       __  __                           _                          
%	 / __ \ / ____|     |  \/  |                         | |                         
%	| |  | | (___ ______| \  / | __ _ ___ ___ _ __   __ _| |__  _ __ ___   ___ _ __  
%	| |  | |\___ \______| |\/| |/ _` / __/ __| '_ \ / _` | '_ \| '_ ` _ \ / _ \ '_ \ 
%	| |__| |____) |     | |  | | (_| \__ \__ \ | | | (_| | | | | | | | | |  __/ | | |
%	 \___\_\_____/      |_|  |_|\__,_|___/___/_| |_|\__,_|_| |_|_| |_| |_|\___|_| |_|
%	          
%%%%%%%%%%%%%%%%%%%%%%%%%%%%%%%%%%%%%%%%%%%%%%%%%%%%%%%%%%%%%%%%%%%%%%%%%%%%%%%%%%%%%%%%%
% .....
%%%%%%%%%%%%%%%%%%%%%%%%%%%%%%%%%%%%%%%%%%%%%%%%%%%%%%%%%%%%%%%%%%%%%%%%%%%%%%%%%%%%%%%%%

% !TeX spellcheck = de_DE
%%%%%%%%%%%%%%%%%%%%%%%%%%%%%%%%%%%%%%%%%%%%%%%%%%%%%%%%%%%%%%%%%%%%%%%%%%%%%%%%%%%%%%%%%
%	 _____           _      _    _        _              _     _               
%	|  __ \         (_)    | |  | |      | |            | |   | |              
%	| |__) | __ ___  _  ___| | _| |_ __ _| |__  ___  ___| |__ | |_   _ ___ ___ 
%	|  ___/ '__/ _ \| |/ _ \ |/ / __/ _` | '_ \/ __|/ __| '_ \| | | | / __/ __|
%	| |   | | | (_) | |  __/   <| || (_| | |_) \__ \ (__| | | | | |_| \__ \__ \
%	|_|   |_|  \___/| |\___|_|\_\\__\__,_|_.__/|___/\___|_| |_|_|\__,_|___/___/
%				   _/ |                                                        
%				  |__/                                                                          
%%%%%%%%%%%%%%%%%%%%%%%%%%%%%%%%%%%%%%%%%%%%%%%%%%%%%%%%%%%%%%%%%%%%%%%%%%%%%%%%%%%%%%%%%
% .....
%%%%%%%%%%%%%%%%%%%%%%%%%%%%%%%%%%%%%%%%%%%%%%%%%%%%%%%%%%%%%%%%%%%%%%%%%%%%%%%%%%%%%%%%%

\chapter{Noch keine Chaptereinteilung vorhanden!}

%\input{./pages/appendix/AppendixProChapter/ZSX}
	\cleardoublepage
	\ifthenelse{\boolean{DEBUG}}{}{\cleardoublepage}
	
	%—————————————————————————————————————————————————————————————————————————
	%						Index
	%—————————————————————————————————————————————————————————————————————————
	\ifthenelse{\boolean{INDEX}}{
		\phantomsection
		\addcontentsline{toc}{chapter}{Index}
		\printindex
	}{}
	\ifthenelse{\boolean{DEBUG}}{}{\cleardoublepage}
	
\end{document}