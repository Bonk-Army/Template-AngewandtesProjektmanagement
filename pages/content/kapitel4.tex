% !TeX spellcheck = de_DE
%%%%%%%%%%%%%%%%%%%%%%%%%%%%%%%%%%%%%%%%%%%%%%%%%%%%%%%%%%%%%%%%%%%%%%%%%%%%%%%%%%%%%%%%%
%	 __  __      _   _               _ _ _    
%	|  \/  |    | | | |             | (_) |   
%	| \  / | ___| |_| |__   ___   __| |_| | __
%	| |\/| |/ _ \ __| '_ \ / _ \ / _` | | |/ /
%	| |  | |  __/ |_| | | | (_) | (_| | |   < 
%	|_|  |_|\___|\__|_| |_|\___/ \__,_|_|_|\_\
%%%%%%%%%%%%%%%%%%%%%%%%%%%%%%%%%%%%%%%%%%%%%%%%%%%%%%%%%%%%%%%%%%%%%%%%%%%%%%%%%%%%%%%%%
%	Hier wird die eigene Arbeit zunächst konzeptuell beschrieben, darunter fällt die Problemanalyse und die 
%	Lösungssuche/-findung. An dieser Stelle soll eine theoretische Auseinandersetzung mit dem Stoff stattfinden – 
%	erklären Sie nicht anhand von konkreten APIs oder Quellcode (Pseudocode hingegen ist erlaubt). Erst danach folgt 
%	die Beschreibung der Realisierung – in den meisten Fällen die Implementierung als Software. Achten Sie hierbei auf 
%	eine übersichtliche und problembezogene Auswahl an Code-Beispielen oder (Teil-)Klassendiagrammen. Detaillierte und 
%	umfassende Darstellungen sollten, wenn überhaupt notwendig, in den Anhang.
%%%%%%%%%%%%%%%%%%%%%%%%%%%%%%%%%%%%%%%%%%%%%%%%%%%%%%%%%%%%%%%%%%%%%%%%%%%%%%%%%%%%%%%%%
\chapter{Methodik} \label{4.methodik}

	\section{Placeholder}
		\lipsum